\documentclass[a4paper, 10pt]{article}
\usepackage[utf8]{inputenc}
\usepackage[spanish]{babel}
\usepackage{graphicx}
\usepackage{geometry}
\usepackage{listings}
\usepackage{amsmath}
\usepackage{amsfonts}
\usepackage{amssymb}
\usepackage{caratula}

\newcommand{\Z}{\mathbb{Z}}
\def\code#1{\texttt{#1}}
\newcommand\tab[1][0.5cm]{\hspace*{#1}}

\geometry{a4paper, margin=0.7in}

\begin{document}
    %Caratula
    \pagenumbering{gobble}
    \newpage

    \begin{center}
        \includegraphics[width=5cm, height=5cm]{images/logo}
    \end{center}

    \materia{Teoría de Algoritmos I}
    \submateria{Primer Cuatrimestre 2017}
    \titulo{Trabajo Práctico 3}

    \integrante{Rodrigo De Rosa}{97799}{rodrigoderosa@outlook.com}
    \integrante{Marcos Schapira}{---}{schapiramarcos@gmail.com}
    \integrante{Facundo Guerrero}{---}{facundoiguerrero@gmail.com}
    \maketitle
    %Fin caratula
    %Table of contents
    \newpage
    \pagenumbering{roman}
    \tableofcontents
    %Fin table of contents
    %Informe
    \newpage
    \pagenumbering{arabic}

    \section{Programación Dinámica}
        \tab En esta sección se analiza una solución al problema de la predcción de acciones a través
        de la programación dinámica.
        \subsection{Algoritmo}
            \subsubsection{Funcionamiento}
                \tab El algoritmo utilizado para resolver el problema planteado funciona de la siguiente
                forma:
                \begin{itemize}
                    \item Determina un \code{día de compra}, un \code{día de venta}, un \code{día de compra auxiliar},
                    una \code{ganancia máxima} y una \code{ganancia temporal}.
                    \item Itera sobre todos los días (valores diferentes de acciones) verificando si
                    en el día actual es más o menos favorable comprar acciones que en el día en el
                    que se pretendía hacerlo hasta el momento, determinando el \code{día de compra auxiliar}.
                    \item A partir del día que determinó, calcula la \code{ganancia temporal} como la que se obtendría
                    si las acciones fueran vendidas el día actual y verifica si es mayor a la \code{ganancia máxima}
                    hasta el momento.
                    \item En tal caso, determina el \code{día de venta} como el actual, el \code{día de compra}
                    como el que previamente era el \code{día de compra auxiliar} y la \code{ganancia máxima} como
                    la que era la \code{ganancia temporal}.
                    \item Al finalizar la iteración, queda determinado el \code{día de compra} más conveniente, el
                    \code{día de venta} más conveniente y la \code{ganancia máxima} obtenible.
                \end{itemize}
            \subsubsection{Ecuación de recurrencia}
                \tab La ecuación de recurrencia del algoritmo utilizado es la siguiente: \\
                \tab\tab\tab $ R(n, m) = ... $ %%Alguna cosa
    \newpage

    \section{Algoritmos Randomizados}
        \tab En esta sección se analiza una solución al problema de hallar el corte global
        mínimo en un grafo no dirigido a través de un algoritmo randomizado.
        \subsection{Algoritmo}
            \tab Para resolver este problema se utilizó el algoritmo de Karger descripto en
            la bibliografía proporcionada por la cátedra.
            \subsubsection{Funcionamiento}
                \tab Este algoritmo %....
            \subsubsection{Categoría de randomización}
                \tab Pertenece a la categoría $X$ porque %....
    \newpage

    \section{Algoritmos Aproximados}
        \tab En esta sección se analiza una solución al problema de la suma de subconjuntos
        a través de un algoritmo aproximado.
        \subsection{Algoritmo}
            \tab Para resolver este problema se utilizó la estrategia polinómica descripta en
            la bibliografía proporcionada por la cátedra.
            \subsubsection{Funcionamiento}
                \tab Este algoritmo %....
    \newpage

    \section{Ejecución de programas}
    \tab Para correr cada algoritmo, se debe ejecutar el archivo principal de cada uno.
    Esto se hace de la siguiente forma: \\
    \tab\tab En la carpeta \code{Programación Dinámica} abrir la consola y ejecutar \code{python main.py} \\
    \tab\tab En la carpeta \code{Algoritmos Randomizados} abrir la consola y ejecutra \code{python main.py} \\
    \tab\tab En la carpeta \code{Algoritmos Aproximados} abrir la consola y ejecutra \code{python main.py} \\

    %Fin informe
\end{document}
